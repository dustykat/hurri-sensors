\documentclass[11pt]{article}
%\documentclass[11pt]{report}
\usepackage{./styles/daves,fancyhdr,natbib,url}
\usepackage{amsmath,amssymb,graphicx,pdfpages,lscape}
\usepackage{booktabs,longtable}
\usepackage{dcolumn}

%%%%%%%%%% Will's stuff BELOW %%%%%%%
\usepackage[left=1.25in, right=1.25in,
            top=1in, bottom=1in]{geometry}                % See geometry.pdf to learn the layout options. There are lots.
\geometry{letterpaper}

\usepackage{ragged2e}

\usepackage{xcolor}
\newcommand{\codeRcolor}{0.93}
\newcommand{\codeGcolor}{0.93}
\newcommand{\codeBcolor}{0.93}
\definecolor{lightgrey}{rgb}{\codeRcolor,
                             \codeGcolor,
                             \codeBcolor}

\newcommand{\listingfont}{\fontsize{7pt}{8pt}\selectfont\ttfamily}
\usepackage{listings}
\lstset{basicstyle = \listingfont,
        breaklines = true,
        frame=tb,
        xleftmargin=12pt,
        framexleftmargin=6pt,
        framexrightmargin=6pt,
        xrightmargin=12pt,
        columns=fixed}
\lstset{lineskip=-1pt}
\lstset{backgroundcolor=\color{lightgrey}}


\usepackage[font={footnotesize},
            labelfont={sf,bf},
            textfont={sf},
            singlelinecheck=false,
            labelsep=none,
            justification=RaggedRight,
            aboveskip=0pt,
            belowskip=7pt plus 1pt minus 1pt,
            textformat=period]{caption}
\DeclareCaptionLabelSeparator{mystyle}{.\quad}
\captionsetup{labelsep=mystyle}
%%%%%%%%% Will's Stuff ABOVE %%%%%%%%%

\cfoot{\thepage}
\lhead{\emph{Database Refinement \mbox{\thinspace ---\thinspace}{Empirical Flow Parameters: A Tool for Hydraulic Model Validity Assessment}}}

\thispagestyle{plain}
%\bibliographystyle{./styles/chicago}
%\bibliographystyle{apalike}

\title{DIY tools for hydrologic monitoring :  An economical way to obtain useable, screening quality data, from ungaged areas.}
\author{P.\ Olar Bear \thanks{Texas Tech University, P.O.~Box 41023, Lubbock, TX 79409}  \\
}
\date{\today}

\begin{document}

\maketitle
%%\nocite{*}

\section*{Abstract}
words

%\tableofcontents


%%%%%%%%%% INTRODUCTION %%%%%%%%%%
\section{Introduction}
The need for data, even low-quality DIY (citizen) collected data.
Ungaged watersheds description.
The economic issue - esp. in re-developing nations.
\section{Literature Review} 
Review of ungaged watersheds methods. \\
Value of obtaining some gage data \\
Rainfall data, the easier of the two \\
Streamflow, the hardest \\
Evaporation, automating the evaporation pan \\
The emergence of low-cost system on a chip (SoC) computers -- a datalogger with an operating system and multi-threading capability.  
Field processing of the raw data. \\
The sensor challenges \\
Analog-to-digital conversion \\
\section{Methods}
words
\subsection{DIY Datalogger using Raspberry Pi SoC Computers}
Review of a handful of SoC computers.  Odroid HC2, Odroid XU4, Arduino, Raspberry Pi (and a few others).\\
The value of an operating system on a datalogger. \\
Raspberry Pi chosen because of availability and support. \\
Programming the sensor-logger interface -- the big decision C++ or Python.\\
Keeping time -- adding a clock for when the computer is disconnected from the internet.\\
\subsection{DIY Sensors}
\subsubsection{DIY Recording Raingage}
Counting events is pressing a button.  Adapting code that detects button presses and using it to read a tipping-bucket gage.
trying to keep this part digital so can run parallel to the analog inputs. Capturing the correct time of the event. \\
\subsubsection{DIY Recording Stage (Depth) Gage}
A resistor-ladder to measure depth.   
How to use 1023 states to get meaningful depth measurements.
An acoustic type gage -- field deployment issues. 
The A/D converter.\\
\subsubsection{DIY Recording Evaporation Gage}
Using a resistor ladder to record evaporation from an evaporation pan.
Issuing a daily control signal to refill the pan to desired level, then stopping the fill.
\section{Prototype Builds}
\section{Prototype Deployment}
\section{Results}
\subsection{Proof-of-Principle}
\subsection{Comparison with a co-located USGS gaging station}
\section{Conclusions and Recommendations}



\begin{thebibliography}{3}

\bibitem[Recking(2010)]{recking2010}
Recking, A., 2010. A comparison between flume and field bed load transport data and consequences for surface-based bed load transport prediction. \textsl{Water Resour. Res., 46}, W03518, doi: 10.1029/2009WR008007.

\bibitem[Peterson(1975)]{peterson1975}
Peterson, A. W., 1975. Universal flow diagrams for mobile boundary channels. \textsl{Can. J. Civ. Eng., 2}, pp 549-557.

\bibitem[Peterson(1973)]{peterson1973}
Peterson, A. W., and Howells, R. F., 1973. A compendium of solids transport data for mobile boundary channels.  Envrion. Can., Inland Waters Dir., Ottawa, Can. 

%\bibitem[Abida and Townsend, 1991]{AbidaTownsend1991}
%Abida, H. and Townsend, R. (1991).
%\newblock Local scour downstream of box-culvert outlets.
%\newblock {\em Journal of Irrigation and Drainage Engineering},
%  117(3):425--441.
%
%\bibitem[Abt et~al., 1984]{Abtetal1984}
%Abt, S., Kloberdanz, R., and Mendoza, C. (1984).
%\newblock Unified culvert scour determination.
%\newblock {\em Journal of Hydraulic Engineering}, 110(10):1475--1479.
%
%\bibitem[Abt et~al., 1985]{Abtetal1985}
%Abt, S., Ruff, J., and Doehring, F. (1985).
%\newblock Culvert slope effects on outlet scour.
%\newblock {\em Journal of Hydraulic Engineering}, 111(10):1363--1367.
%
%\bibitem[Ashida and Michiu, 1972]{AshidaMichiu1972}
%Ashida, K. and Michiu, M. (1972).
%\newblock Study on hydraulic resistance and bed load transport rate in alluvial
%  streams.
%\newblock {\em Transactions, Japan Society of Civil Engineering}, 206:59--69.
%
%\bibitem[Babcock, 1970]{Babcock1970}
%Babcock, H.~A. (1970).
%\newblock The sliding bed flow regime.
%\newblock In {\em 1st International Conference on Hydraulic Transportation of
%  Solids in Pipes}, number Paper H1 in Hydrotransport 1. British Hydromechanics
%  Research Association.
%
%\bibitem[Bathurst et~al., 1987]{bathurstetal1987}
%Bathurst, J.~C., Graf, W.~H., and Cao, H.~H. (1987).
%\newblock Bed load discharge equations for steep mountain rivers.
%\newblock In Thorne, C.~R., Bathurst, J.~C., and Hey, R.~D., editors, {\em
%  Sediment Transport in Gravel-bed Rivers}, pages 453--491. John Wiley \& Sons
%  Ltd.
%
%\bibitem[Biedenharn et~al., 2000]{Biedenharnetal2000}
%Biedenharn, D.~S., Copeland, R.~R., Thorne, C.~R., Soar, P.~J., D.Hey, R., and
%  Watson, C.~C. (2000).
%\newblock Effective discharge calculation: A practical guide.
%\newblock ERDC/CHL TR-00-15, U.S. Army Engineer Research and Development
%  Center, Coastal and Hydraulics Laboratory.
%
%\bibitem[Brownlie, 1981]{brownlie1981}
%Brownlie, W.~R. (1981).
%\newblock Prediction of flow depth and sediment discharge in open channels.
%\newblock Report KH-R-43A, W. M. Keck Laboratory of Hydraulics and Water
%  Resources, California Institute of Technology, Pasadena, California, USA.
%
%\bibitem[Buffington and Montgomery, 1997]{buffingtonandmontgomery1997}
%Buffington, J.~M. and Montgomery, D.~R. (1997).
%\newblock A systematic analysis of eight decades of incipient motion studies,
%  with special reference to gravel-bedded rivers.
%\newblock {\em Water Resources Research}, 33(8):1993--2029.
%
%\bibitem[Chang, 1988]{chang1988}
%Chang, H. (1988).
%\newblock {\em Fluvial Processes in River Engineering}.
%\newblock Krieger.
%
%\bibitem[Chien and Wan, 1999]{ChienWan1999}
%Chien, N. and Wan, Z. (1999).
%\newblock {\em Mechanics of Sediment Transport}.
%\newblock ASCE Press, Reston, Va.
%
%\bibitem[Chiu and Seman, 1971]{ChiuSeman1971}
%Chiu, C.~L. and Seman, J.~J. (1971).
%\newblock Head loss in spiral-liquid flow in pipes.
%\newblock In Zandi, I., editor, {\em Advances in Solid-Liquid Flow in Pipes and
%  its Application}. Pergamon Press Inc.
%
%\bibitem[Church, 2006]{church2006}
%Church, M. (2006).
%\newblock Bed material transport and the morphology of alluvial river channels.
%\newblock 34:32554.
%
%\bibitem[Crookston, 2008]{Crookston2008}
%Crookston, B.~M. (2008).
%\newblock A laboratory study of streambed stability in bottomless culverts.
%\newblock Master's thesis, Utah State University, Logan, Utah.
%
%\bibitem[Crookston and Tullis, 2006]{CrookstonTullis2006}
%Crookston, B.~M. and Tullis, B.~P. (2006).
%\newblock Preliminary study of scour in bottomless culverts.
%\newblock Technical Report FHWA-AK-RD-06-05, Alaska Department of
%  Transportation.
%
%\bibitem[Dietrich et~al., 1989]{Dietrichetal1989}
%Dietrich, W.~E., Kirchner, J.~W., Ikeda, H., and Iseya, F. (1989).
%\newblock Sediment supply and the development of the coarse surface layer in
%  gravel-bedded rivers.
%\newblock {\em Nature}, 340(6230):215--217.
%
%\bibitem[Dodd et~al., 2004]{Doddetal2004}
%Dodd, C.~K., Barichivich, W.~J., and Smith, L.~L. (2004).
%\newblock Effectiveness of a barrier wall and culverts in reducing wildlife
%  mortality on a heavily traveled highway in florida.
%\newblock {\em Biological Conservation}, 118(5):619--631.
%
%\bibitem[Egiazaroff, 1965]{egiazaroff1965}
%Egiazaroff, I.~V. (1965).
%\newblock Calculation of nonuniform sediment concentration.
%\newblock {\em Journal of the Hydraulics Division, Proceedings of the ASCE},
%  91(4):225--248.
%
%\bibitem[Einstein, 1950]{einstein1950}
%Einstein, H.~A. (1950).
%\newblock The bed load function for sediment transport in open channels.
%\newblock Technical {B}ulletin 1026, U.S. Department of Agriculture.
%
%\bibitem[Emmett and Wolman, 2001]{EmmettWolman2001}
%Emmett, W. and Wolman, M. (2001).
%\newblock Effective discharge and gravel-bed rivers.
%\newblock {\em Earth Surface Processes and Landforms}, 26(13):1369--1380.
%
%\bibitem[Fenton and Abbott, 1977]{fentonandabbott1977}
%Fenton, J.~D. and Abbott, J.~E. (1977).
%\newblock Initial movement of grains on a stream bed: the effect of relative
%  protrusion.
%\newblock {\em Proceedings of the Royal Society of London. Series A},
%  352:523--537.
%
%\bibitem[Ferguson and Church, 2004]{fergusonandchurch2004}
%Ferguson, R. and Church, M. (2004).
%\newblock A simple universal equation for grain settling velocity.
%\newblock {\em Journal of Sedimentary Research}, 74(6):933--937.
%
%\bibitem[Garc{\'\i}a, 2008]{asceSedMan2008}
%Garc{\'\i}a, M.~H., editor (2008).
%\newblock {\em Sedimentation Engineering: Processes, Measurements, Modeling,
%  and Practice}.
%\newblock {ASCE} Manuals and Reports on Engineering Practice No. 110. ASCE.
%
%\bibitem[Gilbert, 1960]{Gilbert1960}
%Gilbert, R. (1960).
%\newblock Transport hydraulique et refoulement des mixtures en conduit.
%\newblock {\em Anals des Pontes et Chaussees}, 130(3.4):307--373,437--494.
%
%\bibitem[Goodridge, 2009]{Goodridge2009}
%Goodridge, W.~H. (2009).
%\newblock {\em Sediment Transport Impacts Upon Culvert Hydraulics}.
%\newblock PhD thesis, Utah State University, Logan, Utah.
%
%\bibitem[Graf, 1971]{Graf1971}
%Graf, W.~H. (1971).
%\newblock {\em Hydraulics of Sediment Transport}.
%\newblock McGraw-Hill.
%
%\bibitem[Graf and Acaroglu, 1968]{GrafAcaroglu1968}
%Graf, W.~H. and Acaroglu, E.~R. (1968).
%\newblock Sediment transport in conveyance systems.
%\newblock {\em Bulletin of the International Association of Scientific
%  Hydrology}, 13(2).
%
%\bibitem[Haderlie and Tullis, 2008]{HaderlieTullis2008}
%Haderlie, G. and Tullis, B. (2008).
%\newblock Hydraulics of multibarrel culverts under inlet control.
%\newblock {\em Journal of Irrigation and Drainage Engineering}, 134:507.
%
%\bibitem[Ho and Muste, 2009]{HoMuste2009}
%Ho, H.-C. and Muste, M. (2009).
%\newblock Sedimentation of multi-barrel culverts.
%\newblock In {\em Proceedings of the 2009 Mid-Continent Transportation Research
%  Symposium}, Ames, Iowa.
%
%\bibitem[Hofland and Battjes, 2006]{hoflandandbattjes2006}
%Hofland, B. and Battjes, J.~A. (2006).
%\newblock Probability density function of instantaneous drag forces and shear
%  stresses on a bed.
%\newblock {\em Journal of Hydraulic Engineering}, 132(11):1169--1175.
%
%\bibitem[Hofland et~al., 2005]{hoflandetal2005}
%Hofland, B., Battjes, J.~A., and Booij, R. (2005).
%\newblock Measurement of fluctuating pressures on coarse bed material.
%\newblock {\em Journal of Hydraulic Engineering}, 131(9):770--781.
%
%\bibitem[House et~al., 2005]{Houseetal2005}
%House, M., Pyles, M., and White, D. (2005).
%\newblock Velocity distributions in streambed simulation culverts used for fish
%  passage.
%\newblock {\em Journal of the American Water Resources Association},
%  41(1):209--217.
%
%\bibitem[Howard, 1939]{Howard1939}
%Howard, G.~W. (1939).
%\newblock Transportation of sand and gravel in a four-inch pipe.
%\newblock {\em Transactions ASCE}, 104(2039):1334--1348.
%
%\bibitem[Howard, 1941]{Howard1941}
%Howard, G.~W. (1941).
%\newblock Effects of rifling on four-inch pipe transporting solids.
%\newblock {\em Transactions ASCE}, 106(2101):135--137.
%
%\bibitem[Johnson and Brown, 2000]{JohnsonBrown2000}
%Johnson, P. and Brown, E. (2000).
%\newblock Stream assessment for multicell culvert use.
%\newblock {\em Journal of Hydraulic Engineering-Reston}, 126(5):381--386.
%
%\bibitem[Julien, 1998]{Julien1998}
%Julien, P.~Y. (1998).
%\newblock {\em Erosion and Sedimentation}.
%\newblock Cambridge University Press.
%
%\bibitem[Kerenyi et~al., 2003]{Kerenyietal2003}
%Kerenyi, K., Jones, J., and Stein, S. (2003).
%\newblock Bottomless culvert scour study: Phase 1 laboratory report.
%\newblock Research Report FHWA--RD--02--078, Federal Highway Administration,
%  6300 Georgetown Pike, McLean VA 22101-2296.
%
%\bibitem[Kerenyi et~al., 2007]{Kerenyietal2007}
%Kerenyi, K., Jones, J., and Stein, S. (2007).
%\newblock Bottomless culvert scour study: Phase 2 laboratory report.
%\newblock Research Report FHWA--HRT--07--026, Federal Highway Administration,
%  6300 Georgetown Pike, McLean VA 22101-2296.
%
\bibitem[King et~al., 2004]{kingetal2004}
King, J., Emmett, W., Whiting, P., Kenworthy, R., and Barry, J. (2004).
\newblock Sediment transport data and related information for selected
  coarse-bed streams and rivers in idaho.
\newblock General Technical Report RMRS-GTR-131, USDA Forest Service, Rocky
  Mountain Research Station, Fort Collins, CO.
%
%\bibitem[Lamb et~al., 2008]{lambetal2008}
%Lamb, M.~P., Dietrich, W.~E., and Venditti, J.~G. (2008).
%\newblock Is the critical {S}hields stress for incipient sediment motion
%  dependent on channel-bed slope.
%\newblock {\em Journal of Geophysical Research}, 113:F02008.
%
%\bibitem[Laursen, 1956]{Laursen1956}
%Laursen, E. (1956).
%\newblock The hydraulics of a storm-sewer system for sediment-transporting
%  flow.
%\newblock Technical Report Bulletin 5, Iowa Highway Research Board.

\bibitem[Lee et~al., 2004]{leeetal2004}
Lee, K.~T., Liu, Y.-L., and Cheng, K.-H. (2004).
\newblock Experimental investigation of bedload transport processes under
  unsteady flow conditions.
\newblock {\em Hydrological Processes}, 18:2439--2454.

%\bibitem[Liriano et~al., 2002]{Lirianoetal2002}
%Liriano, S., Day, R., and White, W. (2002).
%\newblock {Scour at culvert outlets as influenced by the turbulent flow
%  structure}.
%\newblock {\em Journal of Hydraulic Research}, 40(3):367--376.
%
%\bibitem[Maxwell et~al., 2001]{Maxwelletal2001}
%Maxwell, A., Papanicolaou, A., Hotchkiss, R., Barber, M., and Schafer, J.
%  (2001).
%\newblock Step-pool morphology in high-gradient countersunk culverts.
%\newblock {\em Transportation Research Record}, 1743(01-2304):49--56.
%
%\bibitem[McEwan and Heald, 2001]{McEwanHeald2001}
%McEwan, I. and Heald, J. (2001).
%\newblock Discrete particle modeling of entrainment from flat uniformly sized
%  sediment beds.
%\newblock {\em Journal of Hydraulic Engineering}, 127(7):588--597.
%
%\bibitem[Meyer-{P}eter and M{\"u}ller, 1948]{meyerpeterandmuller1948}
%Meyer-{P}eter, E. and M{\"u}ller, R. (1948).
%\newblock Formulas for bed-load transport.
%\newblock In {\em Proceedings of the Second Meeting}, pages 39--64, Stockholm,
%  Sweden. IAHR.
%
%\bibitem[Monteith and Pender, 2005]{MonteithPender2005}
%Monteith, H. and Pender, G. (2005).
%\newblock Flume investigations into the influence of shear stress history on a
%  graded sediment bed.
%\newblock {\em Water Resources Research}, 41(12):W12401.
%
%\bibitem[Muste et~al., 2010]{Musteetal2010}
%Muste, M., Ho, H.-C., and Mehl, D. (2010).
%\newblock Insights into the origin and characteristics of the sedimentation
%  process at multi-barrel culverts in iowa.
%\newblock Technical Report IHRB Project TR-596, The Iowa Highway Research
%  Board.
%
%\bibitem[Neill, 1968]{Neill1968}
%Neill, C.~R. (1968).
%\newblock A re-examination of the beginning of movement for coarse granular bed
%  material.
%\newblock Technical Report Report INT 68, Hydraulics Research Station,
%  Wallingford, UK.
%
%\bibitem[Newitt et~al., 1955]{Newittetal1955}
%Newitt, D.~M., Richardson, J.~F., Abbott, M., and Turtle, R.~B. (1955).
%\newblock Hydraulic conveying of solids in horizontal pipes.
%\newblock {\em Transactions Institution of Chemical Engineers}, 33(2):93--113.
%
%\bibitem[Oldmeadow and Church, 2006]{oldmeadowandchurch2006}
%Oldmeadow, D.~F. and Church, M. (2006).
%\newblock A field experiment on streambed stabilization by gravel structures.
%\newblock {\em Geomorphology}, 78:335--350.
%
%\bibitem[Paintal, 1971]{paintal1971}
%Paintal, A.~S. (1971).
%\newblock Concept of critical shear stress in loose boundary open channels.
%\newblock {\em Journal of Hydraulic Research}, 9(1):91--113.
%
%\bibitem[Papanicolaou et~al., 2002]{papanicetal2002}
%Papanicolaou, A.~N., Diplas, P., Evaggelopoulos, N., and Fotopoulos, S. (2002).
%\newblock Stochastic incipient motion criterion for spheres under various bed
%  packing conditions.
%\newblock {\em Journal of Hydraulic Engineering}, 128(4):369--380.
%
%\bibitem[Parker, 1990]{parker1990}
%Parker, G. (1990).
%\newblock Surface-based bedload transport relation for gravel rivers.
%\newblock {\em Journal of Hydraulic Research}, 28(4):417--436.
%
%\bibitem[Parker, 2008]{parker2008}
%Parker, G. (2008).
%\newblock {\em Sedimentation Engineering: Processes, Measurements, Modeling,
%  and Practice}, chapter~3, pages 165--251.
%\newblock {ASCE} Manuals and Reports on Engineering Practice No. 110. American
%  Society of Civil Engineering.
%
%\bibitem[Parker and Klingeman, 1982]{parkerandklingeman1982}
%Parker, G. and Klingeman, P.~C. (1982).
%\newblock On why gravel bed streams are paved.
%\newblock {\em Water Resources Research}, 18(5):1409--1423.
%
%\bibitem[Parker et~al., 1982]{parkeretal1982}
%Parker, G., Klingeman, P.~C., and Mc{L}ean, D.~G. (1982).
%\newblock Bedload and size distribution in paved gravel-bed streams.
%\newblock {\em Journal of Hydraulic Engineering}, 108(4):544--571.
%
%\bibitem[Parker and Sutherland, 1990]{parkerandsutherland1990}
%Parker, G. and Sutherland, A.~J. (1990).
%\newblock Fluvial armouring.
%\newblock {\em Journal of Hydraulic Research}, 28(5):529--544.
%
%\bibitem[Peake, 2004]{Peake2004}
%Peake, S. (2004).
%\newblock An evaluation of the use of critical swimming speed for determination
%  of culvert water velocity criteria for smallmouth bass.
%\newblock {\em Transactions of the American Fisheries Society},
%  133(6):1472--1479.
%
%\bibitem[Peterson and Howells, 1973]{PetersonHowells1973}
%Peterson, A.~W. and Howells, R.~F. (1973).
%\newblock A compendium of solids transport data for mobile boundary channels.
%\newblock Technical report, Inland Waters Directorate, Envrionment Canada.
%
%%\bibitem[Recking, 2010]{Recking2010}
%%Recking, A. (2010).
%%\newblock A comparison between flume and field bed load transport data and
%%  consequences for surface-based bed load transport prediction.
%%\newblock {\em Water Resources Research}, 46(3):W03518.

\bibitem[Recking et~al., 2008a]{reckingetal2008}
Recking, A., Frey, P., Paquier, A., Belleudy, P., and Champagne, J.~Y. (2008a).
\newblock Bed-load transport flume experiments on steep slopes.
\newblock {\em Journal of Hydraulic Engineering}, 134(9):1302--1310.

\bibitem[Recking et~al., 2008b]{reckingetal2008b}
Recking, A., Frey, P., Paquier, A., Belleudy, P., and Champagne, J.~Y. (2008b).
\newblock Feedback between bed load transport and flow resistance in gravel and
  cobble bed rivers.
\newblock {\em Water Resources Research}, 44(5):W05412.

%\bibitem[Schoklitsch, 1949]{schoklitsch1949}
%Schoklitsch, A. (1949).
%\newblock Berechnung der geschiebefracht.
%\newblock {\em Wasser und Energiewirtschaft}, 1.
%
%\bibitem[Shields, 1936]{shields1936}
%Shields, A. (1936).
%\newblock {\em Anwendung der Aehnlichkeitsmechanik und der Turbulenz Forschung
%  auf die Geschiebebewegung}.
%\newblock PhD thesis, Mitt. der Preussische Versuchanstalt f{\"u}r Wasserbau
%  und Schiffbau,, Berlin, Germany.
%
%\bibitem[Strom et~al., 2004]{strometal2004}
%Strom, K., Papanicolaou, A.~N., Evangelopoulos, N., and Odeh, M. (2004).
%\newblock Microforms in gravel bed rivers: Formation, disintegration, and
%  effects on bedload transport.
%\newblock {\em Journal of Hydraulic Engineering}, 130(6):554--567.

%\bibitem[Vanoni, 1975]{asceSedMan1975}
%Vanoni, V.~A., editor (1975).
%\newblock {\em Sedimentation Engineering}.
%\newblock {ASCE} Manuals and Reports on Engineering Practice No. 54. ASCE.
%
%\bibitem[Wargo and Weisman, 2006]{WargoWeisman2006}
%Wargo, R.~S. and Weisman, R.~N. (2006).
%\newblock A comparison of single-cell and multicell culverts for stream
%  crossings.
%\newblock {\em Journal of the American Water Resources Association},
%  42(4):989--995.
%
%\bibitem[Warren and Pardew, 1998]{WarrenPardew1998}
%Warren, M. and Pardew, M. (1998).
%\newblock Road crossings as barriers to small-stream fish movement.
%\newblock {\em Transactions of the American Fisheries Society},
%  127(4):637--644.
%
%\bibitem[Whiting et~al., 1999]{whitingetal1999}
%Whiting, P.~J., Stamm, J.~F., and Moog, D.~B. (1999).
%\newblock Sediment-transporting flows in headwater streams.
%\newblock {\em Geological Society of America Bulletin}, 111(3):450--466.

%\bibitem[Wilcock and Crowe, 2003]{WilcockCrowe2003}
%Wilcock, P.~R. and Crowe, J.~C. (2003).
%\newblock Surface-based transport model for mixed-size sediment.
%\newblock {\em Journal of Hydraulic Engineering}, 129(2):120--128.

\bibitem[Wilcock et~al., 2001]{wilcocketal2001}
Wilcock, P.~R., Kenworthy, S.~T., and Crowe, J.~C. (2001).
\newblock Experimental study of the transport of mixed sand and gravel.
\newblock {\em Water Resources Research}, 37(12):3349--3358.

%\bibitem[Wolman and Miller, 1960]{wolmanandmiller1960}
%Wolman, M.~G. and Miller, J.~C. (1960).
%\newblock Magnitude and frequency of forces in geomorphic processes.
%\newblock {\em Journal of Geology}, 68:54--74.

\bibitem[Wong and Parker, 2006]{wongandparker2006}
Wong, M. and Parker, G. (2006).
\newblock Reanalysis and correction of bed-load relation of meyer-peter and
  m{\"u}ller using their own database.
\newblock {\em Journal of Hydraulic Engineering}, 132(11):1159--1168.

\bibitem[Wong et~al., 2007]{wongetal2007}
Wong, M., Parker, G., DeVries, P., Brown, T.~M., and Burges, S.~J. (2007).
\newblock Experiments on dispersion of tracer stones under lower-regime
  plane-bed equilibrium bed load transport.
\newblock {\em Water Resources Research}, 43:W03440.

\bibitem[Wu and Chou, 2003]{wuandchou2003}
Wu, F.~C. and Chou, Y.~J. (2003).
\newblock Rolling and lifting probabilities for sediment entrainment.
\newblock {\em Journal of Hydraulic Engineering}, 129(2):110--119.

%\bibitem[Yang, 1996]{yang}
%Yang, C.~T. (1996).
%\newblock {\em Sediment Transport: Theory and Practice}.
%\newblock Mc{G}raw-Hill Series in Water Resources and Environmental
%  Engineering. McGraw-Hill.
%
%\bibitem[Zenz and Othmer, 1960]{ZenzOthmer1960}
%Zenz, F.~A. and Othmer, D.~F. (1960).
%\newblock {\em Fluidization and Fluid-Particles Systems}.
%\newblock Reinhold Publishing Corp., New York, NY.
%
%\bibitem[Asquith and Slade(1997)]{AS1997}
%Asquith, W.H., and Slade, R.M., 1997, Regional equations for estimation of peak-streamflow frequency for natural basins in Texas: U.S. Geological Survey Water Resources Investigations Report 96--4307, \url{http://pubs.usgs.gov/wri/wri964307/}.
%
%\bibitem[Asquith, and Roussel (2009)]{AR2009}
%Asquith, W.H., and Roussel, M. S., 2009, Regression equations for estimation of annual peak-streamflow frequency for undeveloped watershed in Texas using an L-moment-based, PRESS-minimized, residual adjusted approach. U.S. Geological Survey Scientific Investigations Report 2009--5087.
%
%\bibitem[Asquith, Herrmann, and Cleveland(2013)]{AsquithQVGAM2013}
%Asquith, W.H., Herrmann, G.R., and Cleveland, T.G., 2013, Use of generalized additive modeling for regionalization of a discharge measurement database in Texas. Journal of Hydrologic Engineering, American Society of Civil Engineers, \textsl{in press}
%%
%\bibitem[Gordon and others(2004)]{Gordon2004}
%Gordon, N.D., T.A. McMahon, B.L. Finlayson, C.J. Gippel, R.J. Nathan, 2004,
%Stream Hydrology: An Introduction for Ecologists (second edition). John Wiley, The Atrium, Southern Gate, Chichester, West Sussex PO19 8SQ,
%England, 429~p.

%
%\bibitem[Homer and other(2004)]{Homer2004}
%Homer, C., Huang, C., Yang, L., Wylie, B., and Coan, M., 2004, Development of a 2001 national land cover database for the United States: Photogrammetric Engineering and Remote Sensing, v.~70, no.~7, p. 829--840, \url{http://www.mrlc.gov/nlcd2001.php}.

%\bibitem[Miller(1953)]{Miller1953}
%Miller, V.C., 1953, A quantitative geomorphic study of drainage basin
%characteristics in the Clinch Mountain area, Virginia and Tennessee: Columbia
%University, Department of Geology, Technical Report, no.~3, Contract N6-ONR
%271--300. [Miller proposed the circular ratio for expressing characteristics of basin shape.]

%\bibitem[Strahler(1952)]{Strahler1952}
%Strahler, A.N. 1952, Dynamic basis of geomorphology: Bulletin of the Geological Society of America, 63:923--938.

%\bibitem[Shreve(1967)]{Shreve1967}
%Shreve, R.L. 1967. Infinite topologically random channel networks: Journal of
%Geology, 75:178--186.


%\bibitem[USGS(2009)]{NWISmeasurements}
%U.S. Geological Survey, 2009, Streamflow measurements for Texas: USGS National Water Information System, accessed March 1, 2009, \url{http://waterdata.usgs.gov/tx/nwis/measurements/?site_no=STATIONID&agency_cd=USGS}.

%\bibitem[USGS(2009b)]{NWISdv}
%U.S. Geological Survey, 2009b, Daily mean streamflow values for Texas: USGS National Water Information System, accessed March 1, 2009, \url{http://waterdata.usgs.gov/tx/nwis/dv/?site_no=STATIONID&agency_cd=USGS}

\end{thebibliography}



\end{document}


